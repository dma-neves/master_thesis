%!TEX root = ../template.tex
%%%%%%%%%%%%%%%%%%%%%%%%%%%%%%%%%%%%%%%%%%%%%%%%%%%%%%%%%%%%%%%%%%%%
%% acknowledgements.tex
%% NOVA thesis document file
%%
%% Text with acknowledgements
%%%%%%%%%%%%%%%%%%%%%%%%%%%%%%%%%%%%%%%%%%%%%%%%%%%%%%%%%%%%%%%%%%%%

\typeout{NT FILE acknowledgements.tex}%

\begin{ntacknowledgements}

% Acknowledgments are personal text and should be a free expression of the author.

% However, without any intention of conditioning the form or content of this text, I would like to add that it usually starts with academic thanks (instructors, etc.); then institutional thanks (Research Center, Department, Faculty, University, FCT / MEC scholarships, etc.) and, finally, the personal ones (friends, family, etc.).

% But I insist that there are no fixed rules for this text, and it must, above all, express what the author feels.

I would like to express my gratitude to everyone who has accompanied and supported me throughout these last academic years.

To start, I would like to thank my thesis advisor Hervé Paulino for all the support and mentorship he has given me during the development of this thesis, but also, for all the interesting research opportunities he has provided me in the last three years.

To the NOVA School of Science and Technology, and in particular the Department of Computer Science, thank you for providing me, and all my colleagues, with an education of excellence, but also for fostering such a great community.

I would also like to thank all of my friends who have made these last five and a half years so enjoyable and memorable, including, but not limited to: Rodrigo Mesquita, Ricardo Valverde, Gonçalo Condeço, Ricardo Monteiro, Rita Costa, Guilherme Gil, Francisco Simões, Joana Paiva, Tomás Santos, André Costa, João Palma, André Matos, Ruben Belo, Guilherme Martins, and João Pio.

Finally, I would like to deeply thank my parents, my brother, my dear Joana, and the rest of my family for all their continuous love and support.

\end{ntacknowledgements}